% ==================================================
%                    Introduction
% ==================================================
It's best to see the introduction as a "triangle" shape in terms of specificity of the subject. You immediately want to start working on describing both the societal and scientific relevance of the paper (usually, but not always in that order). Recently, Nowadays, Since, etc. are easy starters but are heavily overused. Try to make your story more specific with every paragraph, and be sure to have one thread through all of the sentences/paragraphs, it should be a logical flow of zooming in on the material that will eventually bring you to your research question/contributions. While doing so, towards the end, you should be extremely specific about what you are going to do, you set the reader up with required information for grounding the subject, and understanding what your thought process was towards setting up the research, so that they know what to expect. Any citations along the way should include "related but different" work, or some very important papers that gave rise to the work that you are doing here. Make sure that you state a hypothesis (or a research question, alternatively), this helps you think about the experiments required to test this hypothesis. Some papers list the contributions their works offers as bullet points as an ending, this is somewhat of a stylistic choice. The standard questions should at least be clear at the end: what are you doing, why are you doing this, what data are you using, what techniques, what will your experiments demonstrate, what can we learn from that, and so on.
