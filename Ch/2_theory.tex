% ==================================================
%                       Theory
% ==================================================
By now you have introduced all concepts you'll be working with. Time to acknowledge all prior work on this topic, which generally always include a brief historic overview. Note that this includes work on tasks that are similar but not exactly the same as the work that you are doing. Note that this should be related by topic, not technique (as that could blow up very quickly)! A good theory section draws interesting relations and provides a structured overview of the logical progression of all the task's research. You should be able to identify key methodologies, contrasts in approaches, and work towards specifying the research gap you will be addressing with your work. Try to refrain from giving full method section descriptions of all the work that you cite here, typically a core description per piece of related work is enough. Only the very related work deserves a more complete explanation, especially if you will be borrowing some of this methodology. A good strategy for forming an extensive bibliography is to start with the most recent papers on a task, and seeing which are the early papers that they cite. Look those up and see if those cite even earlier work. At some point, you'll hit somewhat of a "beginning" of the field. Try to identify which are the papers with the most impact (say well over 50 citations) and for those with the most impact, look at the "cited by" sections in either google scholar or semantic scholar. This might help you identify papers that are related, but might be difficult to find within the range of queries you can come up with.

\section{Citations} \label{sec:citations}

Citations in parentheses are declared using the \verb|\citep{}| command, and appear in the text as follows: This technique is widely used \citep{woods1970}. The command \verb|\citep{}| (cite parenthetical) is a synonym of \verb|\cite{}|. Citations used in the sentence are declared using the \verb|\citet{}| commands, and appear in the text as follows: \citet{woods1970} first described this technique. The command \verb|\citet{}| (cite textual) is a synonym of \verb|\cite{}|.

Citation commands are based on the \verb|natbib| package; for details on options and further variants of the commands, see the \verb|natbib| documentation.
